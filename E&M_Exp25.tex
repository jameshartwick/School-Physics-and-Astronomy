%%%%%%%%%%%%%%%%%%%%%%%%%%%%%%%%%%%%%%%%%
% University/School Laboratory Report
% LaTeX Template
% Version 3.1 (25/3/14)
%
% This template has been downloaded from:
% http://www.LaTeXTemplates.com
%
% Original author:
% Linux and Unix Users Group at Virginia Tech Wiki 
% (https://vtluug.org/wiki/Example_LaTeX_chem_lab_report)
%
% License:
% CC BY-NC-SA 3.0 (http://creativecommons.org/licenses/by-nc-sa/3.0/)
%
%%%%%%%%%%%%%%%%%%%%%%%%%%%%%%%%%%%%%%%%%

%----------------------------------------------------------------------------------------
% PACKAGES AND DOCUMENT CONFIGURATIONS
%----------------------------------------------------------------------------------------

\documentclass{article}

\usepackage[version=3]{mhchem} % Package for chemical equation typesetting
\usepackage{siunitx} % Provides the \SI{}{} and \si{} command for typesetting SI units
\usepackage{graphicx} % Required for the inclusion of images
\usepackage{natbib} % Required to change bibliography style to APA
\usepackage{amsmath} % Required for some math elements 
\usepackage{amssymb}

\setlength\parindent{0pt} % Removes all indentation from paragraphs

\renewcommand{\labelenumi}{\alph{enumi}.} % Make numbering in the enumerate environment by letter rather than number (e.g. section 6)

%\usepackage{times} % Uncomment to use the Times New Roman font

%----------------------------------------------------------------------------------------
% DOCUMENT INFORMATION
%----------------------------------------------------------------------------------------

\title{Lab 25 \\ Mutual Inductance \\ PHYS 216} % Title

\author{James \textsc{Hartwick}\\Partners: Arianna, Riley} % Author name

\date{Nov 28, 2014} % Date for the report

\begin{document}

\maketitle % Insert the title, author and date

\begin{center}
\begin{tabular}{l r}
Date Performed: & Nov (21, 28), 2014 \\ % Date the experiment was performed
\end{tabular}
\end{center}

% If you wish to include an abstract, uncomment the lines below
% \begin{abstract}
% Abstract text
% \end{abstract}

%----------------------------------------------------------------------------------------
% SECTION 1
%----------------------------------------------------------------------------------------

\section{Objective}
To study the effects of mutual inductance in simple RL circuits.
\section{Theory}
\subsection{The Inductance of Single Coils}
Similarly to how capacitors store energy using the electric field of their charge, inductors store energy in the magnetic field generated by the current flowing through them. In a RL circuit the potential stored will gradually increase with time when a potential is supplied across it. The potential can be expressed as:\\

$\epsilon_E=\epsilon_{FG}\frac{R_E}{R}(1-e^{\frac{-Rt}{L}})$ Equation 25.4\\

Where R is the sum of all resistive elements in the circuit, $R_E$ is the resistance of the circuit resistor, and $\epsilon_{FG}$ is the potential applied across the RL circuit by a function generator.

\subsection{Mutual Inductance of Inductors in Series}
The total inductance of a circuit must account for mutual inductance of inductors on each other. The mutual inductance depends on the distance between coils, the axial orientation of the coils, and the direction of current flow within the coils. All of these will affect the magnitude and direction of the local magnetic field each inductor experiences. Inductors work by making use of the local magnetic field which is the superposition of both the affects of the inductor itself and any other local sources. This superposition of the magnetic field from other sources will either weaken or strengthen the inductance of each inductor. The relationship between two inductors in series can be expressed as:\\

$M=\frac{L_{S2}-L_{S1}}{4}$ Equation 25.9\\

Where M is the mutual inductance, $L_{Sn}$ is the equivalent inductance through each inductor.

\subsection{Inductors in Parallel}
The equivalent inductance of a pair of inductors in parallel may be expressed as:\\

$L_p=(L_1L_2-M^2)/(L_1+L_2\pm2M)$ Equation 25.10\\

Where the 2M term is positive for one inductor and negative for the other. 
\section{Apparatus}
\begin{tabular}{ll}
Two multiturn coils\\
Digital Multimeter\\
LabPro data Acquisition System\\
Function Generator\\
Resistors\\
Computer\\
Leads\\
\end{tabular}

\section{Diagram}

\ 

\ 

\ 

\ 

\ 

\ 

\ 

\ 

\ 

\ 
\ 

\ 
\section{Procedure}
Before starting assembly of the circuit the inductance and resistance of both coils was measured and recorded as well as the resistance of the function generator. Next the function generator was set to produce a square wave at a frequency of 10Hz with a low amplitude and 0V offset.\\

Next the LoggerPro software was set to Repeat mode for a live display. The Experiment length, sampling rate, and triggering settings were adjusted to match observations accordingly.
\subsection{Measuring the Inductance of Single Coils}
The first coil, $L_1$ was connected to the circuit with a resistor. The reference and signal clips were then connected across the resistor as show in Fig 25.5 of the lab manual. The settings in LoggerPro were then adjusted to acquire a waveform to which an automatic curve fit was applied. The graph was then printed and the fit parameter was then used to calculate an experimental value for $L_1$. $L_1$ was then disconnected and the same procedure was repeated for $L_2$.
\subsection{Measuring the Inductance of Inductors in Series}
$L_1$ was then added back into the circuit in series with $L_2$ ensuring that they were coaxial and as close as possible to each other. The same procedure from the previous section was then repeated. The resulting value is either $L_{S1}$ or $L_{S2}$. The direction of the coil windings is not clear so the connections to the coils were reversed and the procedure was again repeated. The greater of the two results will be $L_{S2}$.\\

Next to demonstrate the affect of changes to the physical geometry of the circuit the coils were separated by 3cm and the above procedure was again repeated for find $L_{S1,2}$.
\subsection{Inductors in Parallel}
The two coils were then reconnected in parallel and positioned co-axially and in contact with each other. The same measurement procedure for finding $L_{1,2}$ was then repeated.
\ 
\ 

\ 
\ 

\ 
\ 

\ 

\ 
\section{Experimental Data}
$L_1=(289.4\pm0.05)mH$\\
$R_{L1}=(55.1\pm0.5)\Omega$ (See discussion for details on this uncertainty)\\
$L_2=(97.9\pm0.05)mH$\\
$R_{L2}=(20.4\pm0.5)\Omega$ (See discussion for details on this uncertainty)\\
$R_{FG}=50\Omega$\\
$R_{E}=(102.5\pm0.05)\Omega$
\section{Experimental Analysis}
\subsection{Measuring the Inductance of Single Coils}
Using the fit parameters from Graph 1-2 and equation 25.4.\\
$L_1=\frac{(55.1+102.5+50)\Omega}{701.9}=295.8mH$\\

Uncertainties:\\
$\sqrt{(0.5/55.1)^2+(0.5/50)^2+(0.05/102.5)^2+(0.02)^2}=0.0332$\\
$\implies L_1=(295.8\pm9.8)mH$\\

Similarly:\\
$L_2=(100.8\pm3.34)mH$\\

Consistency checks:\\
$L_1\implies (295.8-289.4)=6.4<(9.8+0.05)=9.85\therefore\ Consistent$\\

$L_2\implies (100.8-97.9)=2.9>(3.34+0.05)=3.39\therefore\ Consistent$
\subsection{Measuring the Inductance of Inductors in Series}
Using the fit parameters from graphs 3-6 and equation 25.4.\\
$L_{S1}=\frac{(20.4+55.1+102.5+50)\Omega}{556.4}=383.2mH$\\

Uncertainties:\\
$\sqrt{(0.5/20.4)^2+(0.5/55.1)^2+(0.5/50)^2+(0.05/102.5)^2+(0.02)^2}=0.0344$\\
$\implies L_{S1}=(383.2\pm13.2)mH$\\

Similarly:\\
$L_{S2}=(409.8\pm14.1)mH$\\

Consistency check:\\
From equation 25.5 is follows that:
$L_1+L_2=\frac{L_{S1}+L_{S2}}{2}$\\

Results should be consistent if this evaluate to within our uncertainties.
$\implies (289.4+97.9)=(383.2+409.8)/2\implies 387.3=396.5$\\ 
$\implies$ an absolute difference of 9.2\\
$9.2<(13.2+14.1)/2=13.65 \therefore\ Consistent.$\\

Obtaining mutual inductance from equation 25.9:\\
$M=\frac{409.8-383.2}{4}=6.65mH$\\
$\implies 2M = 13.3mH\implies \frac{2M}{L_1+L_2}=0.01677$\\

2M is NOT negligible in this case.\\

Similarly as before, but with the coils separated by 3cm:\\
$L_{S1}=388.1mH, L_{S2}=404.4mH$\\
$\implies \frac{2M}{L_1+L_2}=0.01035$\\

2M is still NOT negligible in this case, but its effect is diminished.
\subsection{A Thought Exercise}
To create this circuit arrange the inductors in the same way was for the $L_{S1}$ measurement (subtractive mutual inductance.) Then adjust the distance between the two inductors until 1.9H is the equivalent inductance of the circuit.
\subsection{Inductors in Parallel}
Similarly as before using the fit parameters from Graphs 7-8, but with a different equivalent resistance since the inductors are in parallel.\\

$R_{eq}=14.887\Omega\ \implies\ R_{tot}$=167.4\\

Using these new resistance values the same calculations as above yield:\\
$L_{P1}=(71.11\pm2.45)mH, L_{P2}=(79.37\pm2.73)mH$\\
M=2.065mH\\

Using equation 25.10:
$L_{P1}=73.93mH, L_{P2}=72.37mH$\\

Consistency check:\\
$L_{P1}\implies\ 73.93-71.11=2.82>2.45\ \therefore\ Inconsistent$\\
$L_{P2}\implies\ 79.37-72.37=7.00>2.45\ \therefore\ Inconsistent$\\

\section{Discussion}
All results except for the inductors in parallel were deemed to be consistent. Standard uncertainties in measurement and from the LoggerPro software were included. It was noted that the resistance of the coils was not constant. The resistance value fluctuated with a maximum difference of approximately $\pm0.5\Omega$. Unaccounted for uncertainties may have come from the imprecise/approximated equations that were used. The inductors were calculated as a single point source essentially, when really the true effect on the magnetic field would be a superposition of the effects of each individual turn in each coil with respect to each other turn. The equations used will certainly give a good approximation but are by no means exact.
\subsection{Graphs}
Graph 1.\\
The voltage potential vs time is plotted for the circuit with just the $L_1$ inductor.\\

Graph 2.\\
The voltage potential vs time is plotted for the circuit with just the $L_2$ inductor.\\

Graph 3.\\
The voltage potential vs time is plotted for the circuit with $L_2$ and $L_1$ in series. The fit parameter is used to determine $L_{S1}$\\

Graph 4.\\
The voltage potential vs time is plotted for the circuit with $L_2$ and $L_1$ in series after the connections have been reversed. The fit parameter is used to determine $L_{S2}$\\

Graph 5.\\
The voltage potential vs time is plotted for the circuit with $L_2$ and $L_1$ in series and separated by 3cm. The fit parameter is used to determine $L_{S1}$\\

Graph 6.\\
The voltage potential vs time is plotted for the circuit with $L_2$ and $L_1$ in series after the connections have been reversed and they have separated by 3cm. The fit parameter is used to determine $L_{S2}$\\

Graph 7.\\
The voltage potential vs time is plotted for the circuit with $L_2$ and $L_1$ in parallel. The fit parameter is used to determine $L_{P1}$\\

Graph 8.\\
The voltage potential vs time is plotted for the circuit with $L_2$ and $L_1$ in parallel after the connections have been reversed. The fit parameter is used to determine $L_{P2}$\\
\section{Conclusions}
All results were deemed to be consistent except for the Inductors in parallel.
\subsection{Measuring the Inductance of Single Coils}
$ L_1=(295.8\pm9.8)mH\implies\ Consistent$\\
$L_2=(100.8\pm3.34)mH\implies\ Consistent$

\subsection{Measuring the Inductance of Inductors in Series}
$L_{S1}=(383.2\pm13.2)mH\implies\ Consistent$\\
$L_{S2}=(409.8\pm14.1)mH\implies\ Consistent$\\

2M was determine to be a non-negligible factor even when the coils were separated by 3cm.
\subsection{Inductors in Parallel}

$L_{P1}=(71.11\pm2.45)mH\implies\ Inconsistent$\\
$L_{P2}=(79.37\pm2.73)mH\implies\ Inconsistent$\\
M=2.065mH\\


\end{document}

