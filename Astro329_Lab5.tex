%%%%%%%%%%%%%%%%%%%%%%%%%%%%%%%%%%%%%%%%%
% University/School Laboratory Report
% LaTeX Template
% Version 3.1 (25/3/14)
%
% This template has been downloaded from:
% http://www.LaTeXTemplates.com
%
% Original author:
% Linux and Unix Users Group at Virginia Tech Wiki 
% (https://vtluug.org/wiki/Example_LaTeX_chem_lab_report)
%
% License:
% CC BY-NC-SA 3.0 (http://creativecommons.org/licenses/by-nc-sa/3.0/)
%
%%%%%%%%%%%%%%%%%%%%%%%%%%%%%%%%%%%%%%%%%

%----------------------------------------------------------------------------------------
% PACKAGES AND DOCUMENT CONFIGURATIONS
%----------------------------------------------------------------------------------------

\documentclass{article}

\usepackage[version=3]{mhchem} % Package for chemical equation typesetting
\usepackage{siunitx} % Provides the \SI{}{} and \si{} command for typesetting SI units
\usepackage{graphicx} % Required for the inclusion of images
\usepackage{natbib} % Required to change bibliography style to APA
\usepackage{amsmath} % Required for some math elements 
\usepackage{amssymb}

\setlength\parindent{0pt} % Removes all indentation from paragraphs

\renewcommand{\labelenumi}{\alph{enumi}.} % Make numbering in the enumerate environment by letter rather than number (e.g. section 6)

%\usepackage{times} % Uncomment to use the Times New Roman font

%----------------------------------------------------------------------------------------
% DOCUMENT INFORMATION
%----------------------------------------------------------------------------------------

\title{Lab 5 \\ Spectroscopy \\ ASTR 329} % Title

\author{James \textsc{Hartwick}} % Author name

\date{Dec 16, 2014} % Date for the report

\begin{document}

\maketitle % Insert the title, author and date



\section{Objective}
To investigate and plot the radial velocity curve for Beta Cephei using spectroscopy.
\section{Introduction}
The variable star Beta Cephei can be observed to have a periodic change in luminosity and an observable Doppler shift. The observed Doppler shift of the Helium absorbtion line via spectroscopic analysis can be used to derive the apparent radial velocity of the star. The changes in brightness are caused by the expansion and contraction of gases on the star's surface. When the star radiates upon Helium it expands and appears dimmer due to the higher opacity of Helium in it's higher ionization states comparted to lower ones. The gas then cools as it expands, becoming less opaque to the stars radiation which increases the stars apparent luminosity. This is a cyclical process with a very consistent period. The He I line occurs at 6678\AA. Arc spectroscopy is used to calibrate this wavelength using a Thorium source heated in Argo gas. This will radiate onto the CCD creating a Th-Ar arc which can be used to calibrate the wavelength of the source spectrum.

\section{Procedure}
\subsection{Data Reduction}
The data was first reduced by modifying a IRAF script written by Dr. S. Yang. This is a general use script so it must be modified to work with the Beta Cephei data we are working with. The script automates the following actions.\\

Bias frames are combined and the floating bias is removed. Cosmic ray spikes are filtered out as well.\\

Flat frames are combined and the floating bias is removed. Again cosmic ray spikes are filtered out.\\

Th-Ar arc files are averaged coresponding to the data set.\\

Stellar frames are corrected for floating bias, bias, and dark frames.\\

$flat1d$ is used to do flat field corrections. Each row on the CCD is treated as a spectrum which removes the problem of getting near saturated rows in the flat fielded spectra caused by the division of small numbers.\\

spectra are extracted from 2D to 1D using a upper and lower bound of 25 for standard deviation rejection.\\

\subsection{Data Processing}
Once the reduction script had completed the following steps were followed:\\

The IRAF function $continuum$ was used to fit a continuum to each spectra which the spectra were then normaled with.\\

The $identify$ function was then used on one file using the line list from linelists\$that.dat in epar.\\

$reidentify$ was then run using the previously identified arc as a reference for all other arcs.\\

$refspec$ was then run for individual stellar spectra corresponding to their appropriate arc spectra wavelength references.\\

$dispcor$ was then run to interpolate and select the individual spectral region where the He I line resides.\\

$fxcor$ was then used to get the relative radial velocities among the spectra.\\
\\
\\
\\
\subsection{Analysis}
\subsubsection{Continuum Fitting}
\begin{figure}
\includegraphics[width=1.0\textwidth]{2} % Include the image placeholder.png
\caption{Continuum}
\end{figure}
Once reduce15.cl had completed running the reduced data was normalized with a continuum fit. The $continuum$ command in IRAF transformed the raw reduced spectra into a level absorption spectra. This normalization by the continuum fit ensures that the same parameters are used to fit every image. We can do this because of the assumed similarity of all our images. 
\subsubsection{Identification of Features}
\begin{figure}
\includegraphics[width=1.0\textwidth]{1} % Include the image placeholder.png
\caption{Spectra correlated with wavelength}
\end{figure}
Next the $identify$ function was applied to one of Th-Ar arc files that was produced in the reduction process resulting in the above wavelength dependant spectra with identified lines.

\subsubsection{The He I Absorbtion Line}
\begin{figure}
\includegraphics[width=1.0\textwidth]{5} % Include the image placeholder.png
\caption{He I Absorbtion Line}
\end{figure}
\subsubsection{Relative Radial Velocity Curve}
\begin{tabular}{|c|c|c|c|c|c|}
\hline
Julian Date & FWHM & Rel Vel. (km/s) & Obs Vel. (km/s) & Hel Vel. (km/s) & Error (km/s)\\
\hline
546927.6687 & 108.6800 & 3.4730 & -1.4780 & 3.4720 & 0.9370\\
546927.6720 & 108.7300 & 0.4870 & -4.4630 & 0.4850 & 0.8780\\
546927.7225 & 107.7700 & -20.3130 & -25.2630 & -29.3540 & 0.9080\\
546927.7615 & 105.8500 & -9.5090 &-14.4590 &-9.5800 & 0.6600\\
546927.7872 & 104.4600 & 4.7910 & -0.1600 & 4.7000 & 0.6420\\
546927.8037 & 105.4800 & 12.5110 & 7.5600 & 12.4070 & 0.9790\\
546927.8333 & 107.3100 & 14.3030 & 9.3520 & 14.1790 & 1.0460\\
546927.8486 & 105.6100 & 7.1690 & 2.2190 & 7.0360 & 1.1470\\
546927.8784 & 107.7400 & -8.4880 & -13.4390 & -8.6380 & 1.0380\\
546927.9151 & 106.8500 & -20.9490 & -25.8990 & -21.1140 & 0.6480\\
546927.9470 & 104.6300 & -11.5320 & -16.4830 & -11.7080 & 1.2410\\
546927.9693 & 104.9000 & -0.1310 &-5.0810 & -0.3110 & 1.1980\\
546927.9968 & 105.9300 & 13.5840 & 8.6340 & 13.4030 & 1.0710\\
546928.0227 & 106.9800 & 14.6880 & 9.7380 & 14.5060 & 0.9880\\
546928.0581 & 107.1500 & -2.9650 & -7.9150 & -3.1420 & 1.0430\\

\hline
\end{tabular}\\

Table 1. Data aquired using $dispcor$. The velocities all clearly vary with time as we would expect for Beta Cephei. 
\subsubsection{Telluric Lines}



\begin{tabular}{|c|c|c|c|c|c|}
\hline
Julian Date & FWHM & Rel Vel. (km/s) & Obs Vel. (km/s) & Hel Vel. (km/s) & Error (km/s)\\
\hline
2456927.67 & 28.61 & 0.45 & -4.5 & 0.45 & 0.19\\
2456927.67 & 28.8 & 0.51 & -4.44 & 0.51 & 0.16\\
2456927.72 & 29.5 & 0.06 & -4.89 & 0.02 & 0.07\\
2456927.76 & 29.09 & 0.06 & -4.89 & -0.01 & 0.03\\
2456927.79 & 29.72 & 0.36 & -4.59 & 0.27 & 0.12\\
2456927.8 & 29.52 & -0.03 & -4.98 & -0.13 & 0.07\\
2456927.83 & 29.5 & 0.1 & -4.85 & -0.02 & 0.18\\
2456927.85 & 29.42 & 0.07 & -4.88 & -0.06 & 0.04\\
2456927.88 & 29.42 & -0.09 & -5.04 & -0.24 & 0.33\\
2456927.92 & 29.32 & -0.36 & -5.31 & -0.53 & 0.13\\
2456927.95 & 29.06 & -0.09 & -5.04 & -0.27 & 0.15\\
2456927.97 & 29.31 & -0.28 & -5.23 & -12.94 & 0.11\\
2456928.00 & 29.72 & -0.18 & -5.13 & -0.37 & 0.08\\
2456928.02 & 29.87 & -0.23 & -5.18 & -0.42 & 0.08\\


\hline
\end{tabular}\\

Table 2. Telluric $O_2$ line data. The Telluric lines originate from the earth's atmosphere and thus they can be used to correct for systematic wavelength errors in the He I lines because they are independent of each other.
\begin{figure}
\includegraphics[width=1.0\textwidth]{4} % Include the image placeholder.png
\caption{The Telluric Lines.}
\end{figure}
The average over all the telluric spectra produced in $dispcor$ was used as the reference in $fxcor.$
\subsubsection{Telluric Subtraction}
The values for the Telluric relative radial velocities were subtracted from the Hel values and the results were plotted as a function of the Julian Date to obtain the radial velocity curve (Figure 5.)
\begin{figure}
\includegraphics[width=0.9\textwidth]{fig1} % Include the image placeholder.png
\caption{Radial Velocity Curve}
\end{figure}
\subsubsection{Curve Analysis}
\begin{figure}
\includegraphics[width=1.1\textwidth]{done} % Include the image placeholder.png
\caption{Radial Velocity Curve Fit}
\end{figure}
After fitting a sin wave curve to this data using the software ``Graphical Analysis'' we obtain Figure 6 and the following results.\\

Period = $\frac{2\pi}{33.02}\ =\ (0.190\pm0.00178)$ Days\\
Amplitude = $(19.64\pm0.924)$km/s\\
Phase = 0.300 Days\\\\\\\\\\
\section{Conclusions}
A precise radial velocity curve was produced from analyzing the He I 6678\AA line of Beta Cephei. The dominant source of error was determined to be from the opacity changes on the surface of due to the variable opacity of Helium when it is at different ionization levels. These radial velocity curves may be used to derive a number of exciting things such as the orbital period and perhaps whether or not very large planets orbit the star.\\

The derive curve yielded a Period of $(0.190\pm0.00178)$ Days with an Amplitude of $(19.64\pm0.924)$km/s. This is consistent with the accepted value for the Period of 0.1904844 Days$^1$
\section{References}
1. Beta Cephei. (2014, December 23). In Wikipedia, The Free Encyclopedia. Retrieved 08:10, January 2, 2015, from http://en.wikipedia.org/w/index.php?title=Beta\_Cephei
\end{document}

