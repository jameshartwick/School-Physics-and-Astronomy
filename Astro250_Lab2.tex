

\documentclass{article}

\usepackage[version=3]{mhchem} % Package for chemical equation typesetting
\usepackage{siunitx} % Provides the \SI{}{} and \si{} command for typesetting SI units
\usepackage{graphicx} % Required for the inclusion of images
\usepackage{natbib} % Required to change bibliography style to APA
\usepackage{amsmath} % Required for some math elements 
\usepackage{amssymb}

\setlength\parindent{0pt} % Removes all indentation from paragraphs

\renewcommand{\labelenumi}{\alph{enumi}.} % Make numbering in the enumerate environment by letter rather than number (e.g. section 6)

%\usepackage{times} % Uncomment to use the Times New Roman font

%----------------------------------------------------------------------------------------
% DOCUMENT INFORMATION
%----------------------------------------------------------------------------------------

\title{Lab 2 \\ Characteristics of a CCD \\ ASTR 250\\ \ \\ } % Title

\author{James \textsc{Hartwick}} % Author name

\date{Date Performed: Sept 22, 2014\\ \ } % Date for the report

\begin{document}

\maketitle % Insert the title, author and date


\section{Objective}
To investigate the precision of measurements made with a Charge Coupled Device (CCD).
\section{Introduction}
A CCD is essentially a grid of easily ionizable material. When photons strike the device electrons are freed and collected into buckets spatially corresponding to each pixel. The buckets are read and amplified by the device to convert the analogue signal to digital. CCDs are not perfect and are subject to quantum inefficiencies. In general the noise for a CCD tends to follow a Poisson distribution, but other factors may change this. Other sources of noise include Cosmic rays which will instantly saturate single pixels. Bias noise comes from the unique zero point for each pixel, and it must be subtracted from all science images. Small pixel to pixel efficiency differentials can be corrected for by taking flat field images. Dark current comes from the thermal ionization of electrons. Thus is is vital that the CCD is kept cold. Read noise is introduced by the analog to digital conversion.
\section{Equipment}
CCD camera system STAR 1 on the Climenhaga 0.5 meter telescope.\\
IRAF data analysis program.
\section{Procedure}
\subsection{Part 1}
First remove the telescope covers, set the gain to 4, and ensure all computers and monitors are on a functioning correctly.
\subsubsection{Bias Frame}
The filter was changed to DC and a 0 second exposure was taken.
\subsubsection{Dark Frame}
A 100 second exposure was then taken with the same filter settings. 
\subsubsection{Dome Flats}
The filter was then set to CC and the dome was illuminated with a lamp. A 0.1 seconds exposure was taken. The exposure time was then doubled repeatedly until the final exposure time was 51.2 seconds.
\subsection{Part 2}
The files were downloaded to the local computer after they had be preprocessed.
\subsubsection{Processing the images}
In IRAF the ccdproc library was setup as described in the lab manual$^1$ then run to remove the bias and flat fields.
\subsubsection{Finding the Mean and Standard Deviation of Counts}
In IRAF the imstat library was used to find the mean and standard deviation of a 20x20pixel$^2$ region of the image. The results were then recorded into Table 2.1.
\subsubsection{Testing the Linearity of the CCD}
A file was created with two columns. The first for counts and the seconds for exposure time. Next the IRAF task graph was used to plot the data using the setup detailed in the lab manual$^1$. The resulting log-log plot was then printed.
\subsubsection{Examining the Noise}
After creating a file containing the standard deviation and mean values for counts an awk scripts was used to convert the ADUs to electrons. A gain of 15 was used. Next the IRAF task graph was used to plot the log of the signal against the log of the error. The resulting plot was then printed after updating it's labels and title.
\subsubsection{Examining the Relative Noise}
Next another plot was created of log(N/S) against log(N). An awk script was used to generate the N/S value.
\subsubsection{Theoretical Noise}
Equation 2.4 was used to find the theoretical noise values.
\subsubsection{Verifying the Gain of the Detector}
Another graph was made plotting $N_{ADU}$ against $\sigma_{ADU}^2$.
\section{Data and Calculations}
See attached data page and graph.
\section{Discussion}
\subsection{Questions}
\subsubsection{Finding the Mean and Standard Deviation of Counts}
The results are lower than what would be expected using a Poisson distribution.
\subsubsection{Testing the Linearity of the CCD}
The results are linear. At higher signal levels the slope will flatten out as the CCD saturates, and at low signal the slop will also flatten due to the bias dominating the signal.
\subsubsection{Examining the Noise}
A gain of 15 is used. The plot generated shows that the noise appears to be linear with signal. This is odd because Poisson noise should follow the square root of signal. This indicates other noise is present. This makes sense when looking at equation 2.4 because readout noise is linear with time. To estimate the read noise the higher noise/signal values are chosen from the graph to compute it. This is done because the signal term will become negligible and the noise will be essentially equal to the total noise at these values. 
\subsubsection{Examining the Relative Noise}
This plot shows how the signal normalized error changes as signal increases. It shows a slight downwards trend. A signal of approximately 15000 photons is needed to get a S/N of 100. According to this Graph 1.3 a S/N corresponds to 1000 counts in ADU at a gain of 15. This means that approximately 15000 photons are needed.
\section{Conclusion}
The various relationships between sources of noise on CCD detectors were investigated after taking and processing images. The linear response as well as saturation points of the CCD was plotted. 
\section{References}
1. Department of Physics and Astronomy, Astronomy 250 Lab Manual, pp. 11 to 23. (University of Victoria: Victoria, BC). July 2011.\\

\end{document}


