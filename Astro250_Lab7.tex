

\documentclass{article}

\usepackage[version=3]{mhchem} % Package for chemical equation typesetting
\usepackage{siunitx} % Provides the \SI{}{} and \si{} command for typesetting SI units
\usepackage{graphicx} % Required for the inclusion of images
\usepackage{natbib} % Required to change bibliography style to APA
\usepackage{amsmath} % Required for some math elements 
\usepackage{amssymb}

\setlength\parindent{0pt} % Removes all indentation from paragraphs

\renewcommand{\labelenumi}{\alph{enumi}.} % Make numbering in the enumerate environment by letter rather than number (e.g. section 6)

%\usepackage{times} % Uncomment to use the Times New Roman font

%----------------------------------------------------------------------------------------
% DOCUMENT INFORMATION
%----------------------------------------------------------------------------------------

\title{Lab 7 \\ Chemical Abundances of Arcturus \\ ASTR 250\\ \ \\ } % Title

\author{James \textsc{Hartwick}} % Author name

\date{Date Performed: Nov 10, 2014\\ \ } % Date for the report

\begin{document}

\maketitle % Insert the title, author and date


\section{Objective}
To familiarize students with stellar spectra and the methods for obtaining chemical abundances of stars.
\section{Introduction}
Spectral lines observed in the SED of a star have been shown to be an excellent way of determining the relative abundance of various metals within the star. From these abundance ratios we can attempt to derive meaning to help build models for galactic chemical evolution as well as chemical tagging. If we assume that the observed SED is a good reflection of the initial composition of a star we can learn about the conditions in which the star was formed. For example the oldest stars in the galaxy are those with the lowest metalicity. Metals can only be formed from the violent processes of other stars, so metal rich stars must have formed as later generation stars. Of particular interest are the spectral lines of $\alpha$ elements and Fe. The $\alpha$/Fe ratio can tell us about the environment in which the stars were formed due to the different processes that create these elements.\\

The strength of the spectral lines may be determined by fitting a Gaussian to the line and determining the FWHM. This will determine the absorption present, but emission must also be accounted for. Available emission transitions depend on the environmental conditions the element is experiencing where are described by the atmospheric parameters of the star. The inverse is also true, from the SED we can determine the atmospheric parameters by fitting models against the observed emission lines. 
\section{Equipment}
The following programs will be used along with the 2005 Kuruez Solar Atlas spectral models$^2$ as well as the Arcturus spectrum from the Arcturus Atlas$^3$.\\

DAOSPEC\\
MARCS\\
MOOG 2010\\
\section{Procedure}
First the required files were downloaded from the link provided.
\subsection{Examining the Spectra and Identifying Features}
The splot library in IRAF was used to display the Solar spectra and the spectra of Arcturus. The two were compared, and the Fraunhofer lines of the Solar spectrum were identified.
\subsection{Measuring Spectral Lines}
Next the lines shown in Table 7.2$^1$ were located and the FWHM was found for each by fitting a Gaussian to each line profile. These results were then recorded into Table 7.2. Multiple measurements were made for each using different fit parameters, and uncertainties were generated by the difference between the two measurements. DAOSPEC was then used to calculate these abundances automatically and the results were compared.
\subsection{Finding the Best-Fit Model Atmosphere}
Next the atmospheric parameters were found using the program MOOG. The spectrum was used in MOOG along with the atmospheric parameter models specified in table 7.4 to generate the slopes corresponding to each model. From this a best fit model atmosphere was determined.
\subsection{Finding Chemical Abundances}
Using the best fit model atmosphere the other chemical abundances were then found using MOOG. The results were then added to Table 7.5/6 and uncertainties estimated. 
\section{Data and Calculations}
See attached data page and graph.
\section{Data and Calculations}
See attached Tables and Graphs.
\section{Discussion}
\subsection{Questions}
\subsubsection{Step 1}
The absorption lines for Arcturus appear to be both stronger and red shifted. Also they clearly have different metalicities. The Fraunhofer lines are all absorption lines visible in both stars. 
\subsubsection{Step 2}
The uncertainty does seem to have some correlation to the strength of the line. The strong lines do match up because they follow a Lorentzian distribution rather than the Gaussian which we are using to fit.
\subsubsection{Step 3}
The FeI and FeII abundances are equal. These just correspond to different ionization states of the same atoms so the should be equal. When the gravity and temperature are changed then slope and abundancies do not change significantly. When the microturbulance is increased the slope increases dramatically and the abundances decrease. The best fitting values are 4200k and 1.3km/s. The slopes are essentially flat and the abundances equal. A uncertainty of 0.05 is attached to the microturbulance as no large difference was observed by making a change of this value.
\subsubsection{Step 4}
The errors from MOOG and in Table 7.5 are systematic in nature.
\subsubsection{Abundance Questions}
1. Arcturus is not as metal rich as the sun. This indicates that it is an old star.\\

2. See attached table 7.6.\\

3. The relative ratio is lower than the sun implying Arcturus is more metal poor.\\

4. The average for $\alpha$ elements is 0.203 implying that there are less abundant than in the sun.\\

5. The Arcturus values do agree with the Milky Way stars.\\

6. The sun is a newer star which had more metals present when it was formed.\\
\section{Conclusion}
Various spectral analysis methods were used to demonstrate the fundamentals of metalicity by comparing an older star, Arcturus, to the new more metal rich Sun.
\section{References}
1. Department of Physics and Astronomy, Astronomy 250 Lab Manual, pp. 55 to 69. (University of Victoria: Victoria, BC). July 2011.\\
2. Gustafsson, B., Edvardsson, B., Eriksson, K., Jorgensen, U.G., Nordlund, A., Ples, B. 2008, A \& A, 486, 951\\
3. Ples, B. \& Lambert, D.L. 2002 , A \& A, 386, 1009
\end{document}

