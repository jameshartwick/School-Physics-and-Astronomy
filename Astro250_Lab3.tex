

\documentclass{article}

\usepackage[version=3]{mhchem} % Package for chemical equation typesetting
\usepackage{siunitx} % Provides the \SI{}{} and \si{} command for typesetting SI units
\usepackage{graphicx} % Required for the inclusion of images
\usepackage{natbib} % Required to change bibliography style to APA
\usepackage{amsmath} % Required for some math elements 
\usepackage{amssymb}

\setlength\parindent{0pt} % Removes all indentation from paragraphs

\renewcommand{\labelenumi}{\alph{enumi}.} % Make numbering in the enumerate environment by letter rather than number (e.g. section 6)

%\usepackage{times} % Uncomment to use the Times New Roman font

%----------------------------------------------------------------------------------------
% DOCUMENT INFORMATION
%----------------------------------------------------------------------------------------

\title{Visual Observations \\ ASTR 250\\ \ \\ } % Title

\author{James \textsc{Hartwick}} % Author name

\date{Date Performed: Oct 1, 2014\\ \ } % Date for the report

\begin{document}

\maketitle % Insert the title, author and date


\section{Objective}
The objective of this laboratory exercise is to introduce the basics of practical astronomy. A telescope will be used to provide an observational survey of various stars, planets, and nebulae.
\section{Introduction}
The earliest astronomers were only able to observe stars within the Galaxy and planets in our solar system. The were able to discern and classify these objects into constellations, stars, and planets. Due to the rotation of the earth and relative viewing direction at night as the earth orbits the sun the visible region of sky is constantly changing. Many ancient cultures were able to make use of these facts to count time, longitude, and season. The visible stars forming groups were seen as constellations and inspiration for many cultural mythos worldwide.
\section{Procedure}
The observations done will depend on what is available for viewing at the current time date and weather and seeing conditions.
\section{Equipment}
\subsection{Telescope Parts}
Primary Mirror: The primary mirror is the surface which captures incoming photons for the telescope. The mirror is shaped in such a way that it focuses the reflected light towards the secondary mirror.\\\\
Secondary Mirror: The secondary mirror is used to redirect light to the eyepiece. It is typically an optically flat mirror placed ahead of the focus of the primary mirror. The eyepiece is then situated such that is at the (now redirected) focus of the primary mirror.\\\\
Eyepiece: The eyepiece is placed near the focal point of the secondary mirror to allow for adjustments to the focus of the image and it's magnification.\\\\
Focuser: Used to make adjustments to the focus of the image.\\\\
Mount: There are two main types of telescope mounts equatorial and alt-azimuth. The mount itself serves as a pivot point for the telescope allowing for the location and tracking of stars as they traverse the sky. The equatorial mount has the advantage of easier and smoother tracking of stars as the earth rotates, but it is not always a practical solution, especially for very large telescopes.\\\\
\subsection{Diagram}
.\\
 \\
 \\
 \\
 \\
 \\
 \\
 \\
 \\
 \\
 \\
In comparison to the human eye a 20cm telescope will be able to gather approximately 400 times more light. The observed flux increase proportionally to the square of the radius (area.) 
\section{Observations}
\subsection{1. The Moon}
The moon was not observed during our session due to it being too bright for the 1.8m telescope we were using.
\subsection{2. Planets}
The planet Saturn was observed.\\
\\
\\
\\
\\
\\
\\

\subsection{Albireo}
Albireo was not visually discernible.
\subsection{Mizar}
Mizar was not visually discernible.
\subsection{Constellations}
The big dipper was the only constellation that was discussed. It is located to the north within the Ursa Major constellation.\\
\\
\\
\\
\\
\\
\\
\\
\\
\\

\subsection{Open Clusters, Globular Clusters, Nebulas, and Galaxys}
All of these can be resolvable into individual stars provided they are close enough and your telescope is large enough. No stars were visually discernible using our telescope. Reasons which could effect the resolvability of stars arise from the existence of dust clouds between the stars and the earth, and the resolution limit of the telescope. If the PSF of a star is small enough due to its distance that it blends with other local stars we will not be able to resolve it from earth.
\subsection{Visible Stars at Different Galactic Latitudes}
I was unable to complete this visually, but the trend of stellar density would be greatest along the Galactic plane, especially towards the centre of the galaxy. Elsewhere the stellar density would be significantly less.
\subsection{Measuring Field of View}
This experiment was not attempted during the lab.
\section{References}
1. Department of Physics and Astronomy, Astronomy 250 Lab Manual, pp. 25 to 27. (University of Victoria: Victoria, BC). July 2011.\\
\end{document}


